\abstract
This paper proposes a novel approach to the \jenga{} analysis problem through the use of squared planar markers, physics simulation, and augmented reality. Current approaches in the area assume tower states with no rotational nor translational block movement, which is adequate for analysis in a lab environment, but not in the real world, where towers are subject to block misalignment. The solution put forward in this project not only provides the user with a robust structural integrity analysis, but also improves on state-of-the-art methods for block detection. Evaluation and testing show that this method can be extended for use with structures that are unlike the standard three-by-eighteen \jenga{} tower, provided they can be described using a library of predefined parts; proving the usefulness and potential for wider applications of the method.

\begin{figure}[p]
    \centering
    \includegraphics[width=.5\linewidth]{images/qr}
    \caption{Video demonstration}
    \label{fig:demoqr}
\end{figure}

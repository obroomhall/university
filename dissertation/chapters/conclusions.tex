%---=---==---===---====---=====---======---=====---====---===---==---=---%
%-                             CONCLUSIONS                              -%
%---=---==---===---====---=====---======---=====---====---===---==---=---%

\chapter{Conclusion}\label{chap:conclusions}
This project has provided a method which is robust to real-world problems, such as the randomness of block translations and rotations created on each player turn, unlike other work in the area, specifically the popular \jenga{} robot described by \citet{jengarobot}. It also makes use of augmented reality which grants the system serious market potential, given the modern day rise of AR systems.

There are several problems with the current state of the system, such as with block approximation, discussed in \cref{blockapproximation}, marker detection (\cref{det}), and low unit test coverage (\cref{unit}), which could all be improved in the continuation of the project, given the state of modularity and maintainability of the system. 

One aspect that could certainly have been worked on more effectively is project scope. I found myself having an ever-growing amount of work throughout the project, namely research and attempted implementation of UcoSLAM, a localisation and mapping method proposed in a paper earlier this year \citep{ucoslampaper}. I feel that if the amount of time spent using this emergent solution was used elsewhere in the project, such as the implementation of the visualisation stage, then the system would be substantially stronger and would better display the market potential of the method.

Despite all this, I am happy with the project as a whole, because I reached all the objectives set in \cref{objectives}, and provided a working system and clear proof of concept of a novel method. Furthermore, the system is built in a maintainable and extendable manner, which means that the project can be picked up by another student or like-minded individual who wishes to progress the knowledge and methods in the area.